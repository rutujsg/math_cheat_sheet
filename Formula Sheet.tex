\documentclass[12pt]{article}
\usepackage{graphicx}
\usepackage{booktabs}
\usepackage{amsmath}
\usepackage[margin=1in]{geometry}
\usepackage{multicol}
\usepackage{setspace}
\usepackage{float}
\usepackage{hyperref}

\DeclareMathOperator{\sech}{sech}
\DeclareMathOperator{\csch}{csch}

\title{Mathematical Formulae and Identities}
\author{Rutuj Gavankar \\ \email}
\newcommand{\email}{
	\centering \href{mailto:rsgavank@iu.edu}{rsgavank.edu}}
\newcommand{\diff}[1]{\frac{\mathrm{d}}{\mathrm{d}#1}}
\newcommand{\der}[1]{\, \mathrm{d}#1}


\newtheorem{theorem}{Theorem}

\begin{document}
\maketitle
\tableofcontents
\newpage

\section{Preliminary Algebra}

\subsection{Coordinate Geometry}
Equation of a line:
$$y = mx + c$$
Equation of a circle:
$${(x-h)}^2 + {(y-k)}^2 = r^2$$
Equation of a parabola:
\begin{align*}
y - k &= a{(x-h)}^2 & x-h = a{(y-k)^2} \\
\end{align*}
Equation of an ellipse:
\begin{align*}
\frac{{(x-h)}^2}{a^2} + \frac{{(y-k)}^2}{b^2} = 1 
\end{align*}
Equation of a hyperbola:
\begin{align*}
\frac{{(x-h)}^2}{a^2} - \frac{{(y-k)}^2}{b^2} = 1 
\end{align*}

\subsection{Binomial Expansion}
$$^nC_k = \frac{n!}{k!(n-k)!} = {n\choose k} $$
Also,
\begin{align*}
	{^nC_0} &= {^nC_n} = 1 \\
	{^nC_1} &= {^nC_{n-1}} = n \\
	{^nC_k} &= {^nC_{n-k}}
\end{align*}
For a binomial expression raised to the power \emph{n}:
$${(x+y)}^n  = \sum_{k=0}^{k=n} {^nC_k} x^{n-k}y^k$$

\section{Trigonometry}

\subsection{Common Values \& the Unit Circle}
\begin{table}[H]
	\centering
	\renewcommand{\arraystretch}{1.5}
	\begin{tabular}{cccccc}
		\toprule
		$\theta ^ \circ$ & $0^ \circ$ & $30^ \circ$ & $45^ \circ$ & $60^ \circ$ & $90^ \circ$ \\ \midrule
		$\theta ^ c$ & $0$ & $\frac{\pi}{6}$ & $\frac{\pi}{4}$ & $\frac{\pi}{3}$ & $\frac{\pi}{2}$ \\ \midrule
		$\sin \theta$ & $0$ & $\frac{1}{2}$ & $\frac{1}{\sqrt{2}}$ & $\frac{\sqrt{3}}{2}$ & $1$ \\
		$\cos \theta$ & $1$ & $\frac{\sqrt{3}}{2}$ & $\frac{1}{\sqrt{2}}$ & $\frac{1}{2}$ & $0$ \\
		$\tan \theta$ & 0 & $\frac{1}{\sqrt{3}}$ & $1$ & $\sqrt{3}$ & N/D \\ \bottomrule
	\end{tabular}
	\caption{\label{standardvals}Commonly used trigonometric values. Refer the \hyperref[unitcircle]{Unit Circle}}
\end{table}
\begin{figure}[ht]
	\centering
	\includegraphics[scale=0.8]{Unit_circle_angles.pdf}
	\caption{\label{unitcircle} Unit Circle}
\end{figure}

\subsection{Pythagorean Identities}

\begin{align*}
	\sin^2x + \cos^2x &= 1 &  \sec^2 x - \tan^2x &= 1 \\
	 \csc^2 x - \cot^2 x &= 1
\end{align*}

\subsection{Double Angle Formulae}
\begin{align*}
\sin{(2x)} &= 2\sin x\cos x & \tan{(2x)} &= \frac{2\tan x}{1 - \tan^2x} \\
\cos{(2x)} &= 2\cos^2x - 1 \\
&= \cos^2x - \sin^2x \\ &= 1 - 2\sin^2x
\end{align*}




\subsection{Sum and Difference Formulae}
\begin{align*}
\sin{(x \pm y)} &= \sin x\cos y \pm \cos x\sin y &
\cos{(x \pm y)} &= \cos x \cos y \mp \sin x \sin y \\
\tan{(x \pm y)} &= \frac{\tan x \pm \tan y}{1 \mp \tan x \tan y}
\end{align*}
\begin{align*}
\arcsin x \pm \arcsin y &= \arcsin{(x\sqrt{1 - y^2}\pm y\sqrt{1 - x^2})} \\ \arccos x \pm \arccos y &= \arccos{(xy \mp \sqrt{(1-x^2)(1-y^2)})} \\
\arctan x \pm \arctan y &= \arctan{(\frac{x\pm y}{1 \mp xy})}
\end{align*}

\subsection{Sum to Product Formulae}
\subsection{Product to Sum Formulae}
\subsection{Inverse Trigonometric Functions}
\begin{align*}
	&\sin^{-1}(-x) =-\sin^{-1}x &   &\cos^{-1}(-x) =\pi-\cos^{-1}x, & &\lvert x\rvert\leq 1 \\
	&\tan^{-1}(-x) =-\tan^{-1}x &   &\cot^{-1}(-x) =\pi-\cot^{-1}x,& &x\in\mathbf{R} \\ 
	&\csc^{-1}x =\sin^{-1}\left(\frac{1}{x}\right) & &\sec^{-1}x =\cos^{-1}\left(\frac{1}{x}\right),& &\lvert x\rvert\geq1 \\ &\cot^{-1}x =\tan^{-1}\left(\frac{1}{x}\right),& & & &x>0 \\ 
	&\cot^{-1}x =\pi+\tan^{-1}\left(\frac{1}{x}\right),& & & &x<0 \\ 
	&\sin^{-1}x+\cos^{-1}x =\frac{\pi}{2},& & &  &\lvert x\rvert\leq1 \\ 
	&\csc^{-1}x+\sec^{-1}x =\frac{\pi}{2},& & & &\lvert x\rvert\geq1 
\end{align*}



\subsection{Complex Numbers}
Euler's Formula:
$$re^{i\theta} = r(\cos\theta + i\sin\theta)$$
De Moivre's Formula:
$${(\cos\theta + i\sin\theta)}^n = \cos{(n\theta)} + i\sin{(n\theta)}$$
Exponential Definition of Trigonometric Functions:
\begin{align*}
\cos{(ix)} &= \frac{(e^x + e^{-x})}{2} & \sin{(ix)} &= i\frac{(e^x - e^{-x})}{2} & \tan{(ix)} &= i\frac{(e^x - e^{-x})}{(e^x + e^{-x})}
\end{align*}
Exponential Definition of Hyperbolic Functions:
\begin{align*}
\cosh{(x)} &= \frac{(e^x + e^{-x})}{2} & \sinh{(x)} &= \frac{(e^x - e^{-x})}{2} & \tanh{(x)} &= \frac{(e^x - e^{-x})}{(e^x + e^{-x})}
\end{align*}
Relationship between hyperbolic and trigonometric functions:
\begin{align*}
\cosh x &= \cos{ix} & \cos x = \cosh{ix} \\
i\sinh x &= \sin{ix} & i\sin{x} = \sinh{ix} 
\end{align*}

\subsection{Hyperbolic Identities}
\begin{align*}
&\cosh^2 x - \sinh^2 x = 1 & &\sech^2x + \tanh^2x = 1 & &\csch^2x + \coth = 1\\
&\sinh{(2x)} = 2\sinh x\cosh x & &\cosh{(2x)} = \cosh^2x + \sinh^2{x} & &\sinh x + \cosh x = e^x \\
\end{align*}

\subsection{Inverse Hyperbolic Functions}
\begin{align*}
\cosh^{-1}x &= \ln{(\sqrt{1 + x^2} + x)} & \sinh^{-1}x &= \ln{(\sqrt{1 + x^2} + x)} \\
\tanh^{-1}x &= \ln{\sqrt{\frac{1 + x}{1 - x}}} \\
&= \frac{1}{2}\ln{\frac{1+x}{1-x}}
\end{align*}

\section{Vector Algebra}
\begin{figure}[H]
	\centering
	\includegraphics[scale=1]{vector_add.pdf}
	\caption{\label{vector_add} Addition of two vectors using Parallelogram Law}
\end{figure}
\subsection{Scalar Product}

\begin{align*}
a \cdot b &= \lvert a \rvert \lvert b \rvert \cos\theta \\
&= \langle a |b \rangle
\end{align*}

If $a = x_1 \hat{\imath} + y_1 \hat{\jmath} + z_1 \hat{k}$ and $b = x_2 \hat{\imath} + y_2 \hat{\jmath} + z_2 \hat{k}$ then, 	 
$$a \cdot b = (x_1 x_2)\hat{\imath} + (y_1 y_2)\hat{\jmath} + (z_1 z_2)\hat{k}$$

If $a \cdot b = 0$ then, $ a \perp b$

\subsection{Vector Product}
\begin{align*}
a\times b &= \lvert a \rvert \lvert b \rvert \sin\theta \\
&=|b\rangle \langle a|
\end{align*}

If $a = x_1 \hat{\imath} + y_1 \hat{\jmath} + z_1 \hat{k}$ and $b = x_2 \hat{\imath} + y_2 \hat{\jmath} + z_2 \hat{k}$ then,
$$
a \times b =  
\begin{vmatrix}
\hat{\imath} & \hat{\jmath} & \hat{k} \\
x_1 & y_1 & z_1 \\
x_2 & y_2 & z_2 \\
\end{vmatrix}
$$

\subsection{Equations of lines, planes and spheres}

Equation of a line:
In Fig.\ref{eqnline}, the vector {\bfseries r} can be written as $\mathbf{r} = \mathbf{a + \lambda b}$
\begin{figure}[H]
	\centering
	\includegraphics[scale=0.75]{line}
	\caption{\label{eqnline} The equation of a line. The vector b is in the direction \emph{AR} and $\mathcal{\lambda}$b is the vector from \emph{A} to \emph{R}.}
\end{figure}

\section{Limits}


\section{Differential Calculus}

\subsection{Derivative Rules and Properties}
\begin{align*}
&\diff{x}
\end{align*}


\subsection{Common Derivatives}
\begin{align*}
&\diff{x}a^x = (\ln a)a^x & &\diff{x}\log_a x = \frac{1}{x\ln a} \\
&\diff{x}\sin x = \cos x & &\diff{x}\cos x = -\sin x & &\diff{x}\tan x = \sec^2 x\\
&\diff{x}\csc x = -\csc x \cot x & &\diff{x}\sec x = \sec x \tan x & &\diff{x}\cot x = -\csc^2 x \\
&\diff{x}\sin^{-1}x = \frac{1}{\sqrt{1 - x^2}} & &\diff{x}\cos^{-1}x = - \frac{1}{\sqrt{1 - x^2}} & &\diff{x}\tan^{-1}x = \frac{1}{1 + x^2} \\
&\diff{x}\sec^{-1}x = \frac{1}{ x \sqrt{x^2 - 1}} &  &\diff{x}\csc^{-1}x = -\frac{1}{ x \sqrt{x^2 - 1}} & &\diff{x}\cot^{-1}x = -\frac{1}{1 + x^2} \\
\end{align*}



\section{Integral Calculus}
\subsection{Fundamental Theorem of Calculus}
\begin{theorem}[First Fundamental Theorem of Calculus]
	\label{ftc}
	If $f$ is continuous on $[a,b]$, then the function defined by
	$$S(x)=\int _{ a }^{ x }{ f(t)\, dt }$$
	is continuous on $[a,b]$ and differentiable on $(a,b)$, and $S'(x)=f(x)$.
\end{theorem}
Written in Leibniz notation,
$$\diff{x}\int_{a}^{x} f(t)\, dt = f(x)$$

\begin{theorem}[Second Fundamental Theorem of Calculus]
	If $f$ is a continuous function on $[a,b]$, then	
	$$\int _{ a }^{ b }{ f(x)\, dx=F(b)-F(a)}$$
	where $F$ is the anti-derivative of $f$, i.e. $F'=f$.
\end{theorem}
\subsection{Common Antiderivatives}
\begin{align*}
&\int\frac{1}{x}\der{x} = \ln|x| + c & &\int{\frac{1}{ax + b}\der{x}} = \frac{1}{a}\ln|ax+b| + c \\
&\int \cos a \der{x}= \frac1a \sin ax + c & &\int \sin ax\der{x}= - \frac1a \cos ax + c \\
&
\end{align*}



\end{document}